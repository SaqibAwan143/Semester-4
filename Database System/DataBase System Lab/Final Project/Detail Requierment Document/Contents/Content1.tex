% Project Description


\addcontentsline {toc} {section} {Project Description}
\section*{Project Description}
\label{Project Description}
\lhead{Project Description}

The task of creating a website for course management is complex and multifaceted. The website is designed to serve three main types of users: Admin, teachers and students. Each user has different permissions and responsibilities designed to create an efficient management system.

The main purpose of this site is for students to apply to the courses and take the final exam consisting of question banks given by the lecturer of the course. The website is designed to provide students with easy access to course materials and resources, as well as a platform where they can communicate with teachers and fellow students.
The website will be developed using modern web development techniques such as \textbf{HTML, CSS, JavaScript, and C Sharp}. These technologies will be used to create user-friendly and responsive interfaces that work equally well on all devices and platforms.

The site will have three main types of users, each with different access and permission levels:

\subsection*{Admin:}
Admin is the highest-level user on the site and is responsible for managing all activities of the site. Admin will be able to add and remove instructors, classes, and students from the system. They can also view and manage student performance in different classes.
\subsection*{Instructor:}
 Instructors are users who can create lessons, add course materials, and submit final question banks. They will also be able to view and evaluate the performance of students enrolled in their classes. Teachers will be able to add or remove students from their classes.

\subsection*{Students:}
 Students are users who can search for different courses on the website and can register for the courses they want. After enrolling, they will have access to study materials and a question bank for the final exam.
After graduation, they will be able to take the final exam. The final exam will consist of a question bank provided by the instructor.

The website will be designed to be user-friendly and easy to navigate. It will have a simple and intuitive interface that is easy to use, even for users unfamiliar with web technology. The site will be divided into several sections, each designed to meet the needs of a specific user group.
For example, the student section of the website will have a search function so that students can quickly find their courses. Students will also have a dashboard where they can view their progress in different classes, view their grades, and interact with other students and teachers.

Teacher's website will have a lesson management system where teachers can create lessons, add lesson materials and send question banks. Teachers can also view and evaluate the performance of students enrolled in their classes and communicate with them via messaging.

Site Admin will have full control, allowing Admin to manage all site activities.
Admin will be able to view and manage all courses, teachers and students in the system. They will also be able to monitor and manage student performance in different classes.

The website will be built to be flexible and adaptable, able to handle more information and traffic. The site will be built using a modular design that will make it easy to add new features and functionality in the future.

In a nutshell, a classroom management website creation project is complex and multifaceted.
The site will be designed to accommodate three main types of users.
\clearpage